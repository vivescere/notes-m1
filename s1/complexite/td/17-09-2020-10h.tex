\section{17 Septembre 2020}

\subsection{Exercice 3}

\subsubsection{Question 1}

\begin{verbatim}
def matrice_produit(a, b):
	for i in range(1, n):
		for j in range(1, n):
			out[i][j] = 0 # C1
			for k in range(1, n):
				out[i][j] += a[i][k] + b[k][j] # C2
	
	return out
			
\end{verbatim}

La taille des données d'entrée = la taille $k$ (bornée par une constante) de codage d'un entier $*n^2$, fois 2 pour car on a 2 matrices.

On pose : $t=2kn^2 \iff n=\frac{t}{2k}^\frac{1}{2}$.

	$temps = n^2(c_1+nc_2)=c_1n^2+c_2n^3=\frac{C_1^t}{2K}+C_2(\frac{t}{2k})^\frac{2}{3} \in \theta(t^\frac{3}{2})$

\subsubsection{Question 2}

Oui.

\subsubsection{Question 3}

\begin{verbatim}
matrice puissance(matriceA, matriceB):
	B=A
	for i = 1 to P-1
		b = produit(B, A)
	retourner B
\end{verbatim}

Taille de l'entée : $t=kn^2+k$

