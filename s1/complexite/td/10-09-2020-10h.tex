\hypertarget{jeudi-10-septembre-2020}{%
\section{Jeudi 10 Septembre 2020}\label{jeudi-10-septembre-2020}}

Planche de TD 1.

\hypertarget{exercice-1}{%
\subsection{Exercice 1}\label{exercice-1}}

Pour savoir si \(g(n) \in O(f(n))\), il suffit de montrer qu'à partir
d'un certain rang \(n_0\), la fonction est majorée par \(O(Cf(n))\),
avec \(C\) une constante :

$$
\exists C \in \\R, \exists n_0 \in \\N, \forall n \geqslant n_0, g(n) \leqslant Cf(n)
$$

\hypertarget{question-1}{%
\subsubsection{Question 1}\label{question-1}}

\begin{itemize}
\item
  \(g(n) = 6n + 12 \in O(n)\)

  On trouve \(C\) et \(n_0\), et on vérifie que
  \(g(n) \leqslant Cf(n)\). Dans ce cas, on peut prendre \(C=7\) et
  \(n_0=12\).

  Puis-ce que \(\forall n \geqslant 12, g(n) < 7n\), alors
  \(g(n) \in O(n)\)
\item
  \(g(n) = 3n \in O(n^2)\)

  \(C=4\), \(n_0=10\)
\item
  \(g(n) = 10^{1000000}n \in O(n)\)

  \(C=10^{1000000}+1\), \(n_0=1\)
\item
  \(g(n) = 5n^2+10n \in O(n^2)\)

  \(C=6\), \(n_0=100\)
\item
  \(g(n) = n^3+1000n^2+n+8 \notin O(n^2)\)

  Par l'absurde:
\end{itemize}

\[
\forall C \in \\R, \forall n_0 \in \\N, \exists n \geqslant n_0, g(n) \geqslant Cf(n)
\]

\[
\iff \forall C \in \\R, \forall n_0 \in \\N, \exists n \geqslant n_0, n^3+1000n^2+n+8 \geqslant Cn^2 
\]

\[
\iff \forall C \in \\R, \forall n_0 \in \\N, \exists n \geqslant n_0, n + 1000 + \frac{1}{n} + \frac{8}{n^2} \geqslant C
\]

$\forall C \in \\N$, il existe
$n > C - 1000 - \frac{1}{n} - \frac{8}{n^2}$

\hypertarget{question-2}{%
\subsubsection{Question 2}\label{question-2}}

\(g(n) \in \theta(f(n))\) si :

\[
\exists C_1, C_2 \in \\R, \forall n \geqslant n_0, C_1f(n) \leqslant g(n) \leqslant C_2f(n)
\]

\begin{itemize}
\item
  \(g(n) = 5n^2 + 10n \in \theta(n^2)\)

  \(5n^2 \leqslant 5n^2 + 10n \leqslant 6n^2\) avec \(n_0 = 1000\)
\item
  \(g(n) = n^2 + 1000000n \in \theta(n^2)\)

  \(n^2 \leqslant n^2 + 1000000n \leqslant 2n^2\) avec
  \(n_0 = 10^{100}\)
\item
  \(g(n) = 4n^2 + n.log(n) \in \theta(n^2)\)

  \(4n^2 \leqslant 4n^2 + n.log(n) \leqslant 5n^2\) avec \(n_0 = 1000\)
\item
  \(g(n) = 3n+8 \notin \theta(n^2)\)

  Il suffit de montrer que
  \(\forall C_1\in\\R, \forall n_0 \in \\N, \exists n \geqslant n_0, C_1n^2 > 3n+8\)

  \(\iff C_1n^2>3n+8 \iff C_1 > \frac{3}{n} + \frac{8}{n^2}\)

  \(\forall n \in \\N^*, \frac{3}{n}+\frac{8}{n^2} \leqslant 11\), donc
  avec \(C_1 > 11\) on a \(\forall n \in \\N^*, C_1n^2 > 3n+8\)
\end{itemize}

\hypertarget{exercice-2}{%
\subsection{Exercice 2}\label{exercice-2}}

\hypertarget{question-1-1}{%
\subsubsection{Question 1}\label{question-1-1}}

\begin{verbatim}
puissance(x, p):
  res = 1
  while p > 0:
    res = res * x
    p = p - 1
  return res
\end{verbatim}

\hypertarget{question-2-1}{%
\subsubsection{Question 2}\label{question-2-1}}

Rappel : \(t_x = taille(x) = \lfloor log(x) \rfloor + 1\)

On a donc \(x = 2^{tx}-1\) et \(p=2^{tp}-1\).

La ligne \texttt{res\ =\ res\ *\ x} est au pire \(\leqslant tx * p\). La
ligne \texttt{p\ =\ p\ -\ 1} est au pire tp opérations.

Ces deux lignes sont donc égales (au pire) à
\(P(tx*p+tp) \iff 2^{tp}(tx*2^{tp}+tp) \simeq 2^{2tp}tx+tp*2^{tp} \in \theta(2^{2tp}tx)\).

Note : on aurait du ajouter le test de la boucle.
