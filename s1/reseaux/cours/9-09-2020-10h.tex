\hypertarget{premier-cours}{%
\section{Premier cours}\label{premier-cours}}

9 septembre 2020, 10h

\hypertarget{informations-utiles}{%
\subsection{Informations utiles}\label{informations-utiles}}

Répartition horaire :

\begin{itemize}
\tightlist
\item
  12h de cours
\item
  12h td
\item
  14h tp (avec une partie projet)
\end{itemize}

Le professeur en CM est Djamal Habet, et le responsable des TPs est
Emmanuel Godard.

Evaluation :

\begin{itemize}
\tightlist
\item
  Session 1: \(\frac{2}{3}ET+\frac{1}{3}TP\)
\item
  Session 2: \(Max(ET, \frac{2}{3}ET+\frac{1}{3}TP)\)
\end{itemize}

Les scéances de TP contiendront un projet.

\hypertarget{introduction-rappels}{%
\subsection{Introduction \& rappels}\label{introduction-rappels}}

L'UE réseau de la L3 se concentrait sur le modèle application, celle de
cette année va plus s'intéresser au fonctionnement du réseau dans sa
globalité.

Contenu du cours :

\begin{itemize}
\tightlist
\item
  Introduction \& notions de base
\item
  Modèle OSI
\item
  Pile TCP/IP
\item
  Applications
\end{itemize}

Note : Un graphe permet de représenter facilement un réseau.

\hypertarget{taille-des-ruxe9seaux}{%
\subsubsection{Taille des réseaux}\label{taille-des-ruxe9seaux}}

Les réseaux ont une taille (géographique) :

\begin{itemize}
\tightlist
\item
  un bus est à une échelle très petite (eg: carte mère)
\item
  un LAN ne va que de quelques m à quelques km
\item
  un MAN (Metropolitan Area Network) est à l'échelle d'une ville
\end{itemize}

par exemple un LAN est local seulement qui va de quelques mètres à
quelques km, un PAN est un réseau personnel, un WAN (Wide Area Network)
est plus large (pays/monde), un bus est très petit (sur une carte mère
par exemple), un MAN (Metropolitan Area Network) est à l'échelle d'une
ville, \ldots{}

\hypertarget{topologie}{%
\subsubsection{Topologie}\label{topologie}}

Très détaillé sur le transparent.

On a plusieurs topologies pour les réseaux, comme : - en diffusion (eg:
bus, anneau, satellite), comme pour la radio; C'est l'adresse spécifique
placée dans le message qui permet de déterminer qui va le recevoir. Ce
mode est très utilié pour les réseaux locaux (eg: en salle de TP). On
peut représenter ce réseau par un arbre. - en point à point (eg: étoile,
boucle simple, boucle double, maillage régulier, maillage irréglier)

On va donc utiliser des algorithms comme dijkstra pour router les
messages.

\hypertarget{communication}{%
\subsubsection{Communication}\label{communication}}

\hypertarget{avec-connexion}{%
\paragraph{Avec connexion}\label{avec-connexion}}

Comme pour le téléphone, l'émetteur demande l'établissement d'une
connexion par l'envoi d'un bloc de données spécial. Si la connexion est
aceptée, elle est établie par la mise en place d'un circuit virtuel.

Exemple: TCP

\hypertarget{sans-connexion}{%
\paragraph{Sans connexion}\label{sans-connexion}}

On appelle les blocs de données datagrammes. Ils sont tranmis sans
vérifier qu'un circuit existe, il suffit juste d'une adresse de
destination.

Exemple: UDP

\hypertarget{commutation}{%
\subsubsection{Commutation}\label{commutation}}

\hypertarget{commutation-de-circuits}{%
\paragraph{Commutation de circuits}\label{commutation-de-circuits}}

Consiste à créer dans le réseau un circuit entre l'émetteur et le
récepteur avant que ceux-ci commencer à échanger des informations.

C'est ce que les opératrices au tout début du téléphone faisaient.

\hypertarget{commutation-de-messages}{%
\paragraph{Commutation de messages}\label{commutation-de-messages}}

Consiste à envoyer un message de l'émetteur jusqu'au récepteur en
passant de noeud de commutation en noeud de commutation.

On risque d'avoir des problèmes avec les messages longs. -\textgreater{}
taux d'erreurs résiduel

\hypertarget{commutation-de-paquets}{%
\paragraph{Commutation de paquets}\label{commutation-de-paquets}}

Consiste à découper un message en paquets avant de l'envoyer. On éviter
ainsi les erreurs avec les messages longs.

Problème: il faut que le récepteur final sache reconstituer le message,
et donc il faut un protocole particulier.

On va devoir avoir plusieurs champs dans chaque paquet: - un ID de
message - un nombre pour l'ordre - un booléen pour savoir si c'est le
dernier paquet

\hypertarget{commutation-des-cellules}{%
\paragraph{Commutation des cellules}\label{commutation-des-cellules}}

Pareil que la commutation de paquets, avec une taille fixée de paquet de
53 octets (5 d'en-tête et 48 de données).

La faible taille, permet de limiter les erreurs.

\hypertarget{la-normalisation}{%
\subsubsection{La normalisation}\label{la-normalisation}}

-\textgreater{} distinction norme/standard, une norme est étable par un
organisme donc c'est officiellement le rôle alors qu'un standard est
rédigé par une entité non reconnue -\textgreater{} rôle des organismes
de normalisation: définir un cadre, souvent nommé modèle, et aussi
garantir la complétude \& l'intégrité des spécifications -\textgreater{}
organismes les plus connus: ISO, UTI (ex CCITT), IEEE

\hypertarget{le-moduxe8le-osi}{%
\paragraph{Le modèle OSI}\label{le-moduxe8le-osi}}

=\textgreater{} Operated Open System

Architecture en couche Double objectif: - avoir à remplacer seulement le
ou les modules nécessaires - se procurer les modules chez différents
fournisseurs

insérer schéma en U du modèle OSI

à l'envoi d'un message, on le fait passer par toutes les couches. Chaque
couche ajoute sont en-tête (la couche laison ajoute aussi à la fin du
message).

à la réception, on fait l'inverse.

On a 7 couches: - application 7 - présentation 6 - session 5 - transport
4 - réseau 3 - liaison 2 - physique 1

Les 4 premières couches sont orientées réseau (en général faites par
l'OS), les 4 dernières sont orientées application.

Attention: ce modèle est \textbf{théorique}. On peut se permettre de
changer les choses un peu.

Par exemple, les couches physique et les couches laison sont un peu
liées (pas totalement séparées).

\hypertarget{la-couche-physique}{%
\subparagraph{La couche physique}\label{la-couche-physique}}

Elle fournit les moyens mécaniques, électriques, fonctionnels et
procédurax nécessaires à l'éactivation des connexions physiques
destinnées à la ttransmission de bits entre deux entités de laison de
données.

=\textgreater{} On transmet du binaire.

Sens de transmission

-\textgreater{} mode simplex (eg: radio) -\textgreater{} mode
semi-duplex (eg: talkie walkie) -\textgreater{} mode duplex (eg:
websocket)

Transmission parallèle

Les bits d'un même caractère sont envoyés en même temps chacun sur un
fil distinct.

Pose des problèmes de synchronisation et n'est utilisé que sur des
courtes distances (eg: bus).

Transmission en séérie

Les bits sont envoyés les uns derrière les autres de manière synchrone
ou asynchrone.

Synchrone -\textgreater{} l'émetteur et le récepteur se mettent d'accord
sur une base detemps régulier. Asynchrone -\textgreater{} pas de
négociations, chaque caractère est précédé d'un bit de start et suivi
d'un bit de stop. de nos jours, un seul bit de suffit plus

\hypertarget{duxe9bits}{%
\subparagraph{Débits}\label{duxe9bits}}

Quel que soit le monde de transmission, on peut compter le débit de la
ligne en Bauds : le nombre de tops d'horloge en une seconde.

Le nombre de bauds correspond au nombre d'émissions autorisés, donc:
Bauds = Bits/S

Si le signal prend \(2^n\) valeurs distinctes, on dit que sa valence est
de n.
