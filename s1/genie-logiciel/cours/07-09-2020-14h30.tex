\section{Second cours}

7 septembre 2020, 14h30

\hypertarget{uxe9tude-de-cas}{%
\subsection{Étude de cas}\label{uxe9tude-de-cas}}

Cette partie présente une étude de cas sur un système de commande pour
le chargement automatique d'un chariot.

Nous ne réalisons pour le moment que la spécification.

\hypertarget{uxe8re-phase-moduxe9lisation-de-lenvironnement}{%
\subsubsection{1ère phase : Modélisation de
l'environnement}\label{uxe8re-phase-moduxe9lisation-de-lenvironnement}}

On commence par déterminer et caractériser les objets de
l'environnement. On peut ensuite les classer en différentes catégories :
données, activités et relations. C'est dans cette situation que l'on
pourrait par exemple utiliser UML.

Dans notre exemple, on a 4 objets :

\begin{itemize}
\tightlist
\item
  le système de contrôle-commande (notre ``programme'')
\item
  dans l'environnement:

  \begin{itemize}
  \tightlist
  \item
    l'opérateur
  \item
    le chariot
  \item
    le tapis chargeur
  \end{itemize}
\end{itemize}

On utilise 3 variables à états discrets :

\begin{itemize}
\tightlist
\item
  P (position) : \(P \leqslant P2\), \(P2 < P < P1\),
  \(P \geqslant P1\)
\item
  V (vitesse) : \(+V\) (vers la droite), \(-V\) (vers la gauche), 0
\item
  T (trappe) : ouverte, fermée
\end{itemize}

On note :

\begin{itemize}
\tightlist
\item
  V dépend de la commande de déplacement appliquée : Avancer, Reculer,
  Arrêter
\item
  T dépend de la commande d'ouverture appliquée : Ouvrir, Fermer
\end{itemize}

On peut alors créer un automate, ici pour la trappe :

\begin{center}
	\begin{tikzpicture}[shorten >= 1pt,node distance=4cm,on grid,auto]
		\node[state] (ouverte) {Ouverte};
		\node[state] (fermée) [left=of ouverte] {Fermée};
		\path[->] 
		(ouverte) edge [bend left, above] node {Fermer} (fermée)
		edge [loop above] node {Ouvrir} ()
		(fermée) edge [bend left, above] node {Ouvrir} (ouverte)
		edge [loop above] node {Fermer} ();
		\draw[<-] (fermée) -- node[above] {} ++(-1.5cm,0);
	\end{tikzpicture}
\end{center}

Ou alors pour la vitesse du chariot :

\begin{center}
	\begin{tikzpicture}[shorten >= 1pt,node distance=4cm,on grid,auto,state/.style={circle, draw, minimum size=1.5cm}]
		\node[state] (p2) {$P \leqslant P2$};
		\node[state] (depd) [above right=of p2] {$\rightarrow$};
		\node[state] (depg) [below right=of p2] {$\leftarrow$};
		\node[state] (p1) [below right=of depd] {$P > P2$};
		\draw[<-] (p2) -- node[below] {} ++(-1.5cm,0);
		\path[->]
		(p2) edge node {$+V$} (depd)
		(depd) edge node {$\emptyset$} (p1)
		(p1) edge [bend right, above] node {$-V$} (depg)
		(depg) edge node {$\emptyset$  ($P \leqslant P2$)} (p2)
		edge [bend right, above] node {$\emptyset$ ($P > P2$)} (p1);
	\end{tikzpicture}
\end{center}

Avec $\leftarrow$ un déplacement à gauche, et $\rightarrow$ un déplacement à droite. Attention, on n'est pas forcément au point $P1$ à l'état $P > P2$, juste à l'arrêt à droite du point $P2$.

\hypertarget{nd-phase-duxe9termination-des-entruxe9es-et-des-sorties}{%
	\subsubsection{2\textsuperscript{e} phase : détermination des entrées et des
sorties}\label{nd-phase-duxe9termination-des-entruxe9es-et-des-sorties}}

On spécifie le système qui gère les entrées et les sorties de chaque
objet, c'est-à-dire les commandes que l'on envoie aux objets.

\begin{figure}[H]
	\centerfloat
	\resizebox{1.3\textwidth}{!}{%
	%\makebox{\pagewidth}
	\begin{tikzpicture}[shorten >= 1pt,node distance=4cm,on grid,auto,state/.style={circle, draw, minimum size=2.5cm}]
		\node[state] (p2) [align=center] {à l'arrêt \\ $P2$};
		\node[state] (vidage) [below right=of p2] [align=center] {vidage \\ chariot};
		\node[state] (depd) [above right=of vidage, align=center] {chariot \\ se déplace};
		\node[state] (depg) [below right=of depd, align=center] {déplacement \\ à gauche};
		\node[state] (p1) [above right=of depg, align=center] {chariot \\ à l'arrêt};
		\path[->]
		(p2) edge node [align=center] {cycle\\+V} (depd)
		(depd) edge node [align=center] {capteur droit / $\emptyset$\\mettre en marche} (p1)
		(p1) edge node [align=center] {$temps \geqslant temps\_remplissage$ \\ mettre\_à\_l'arrêt \\ -V} (depg)
		(depg) edge node [align=center] {capteur gauche / $\emptyset$\\ouvrir} (vidage)
		(vidage) edge node [align=center] {$temps \geqslant$ temps\_vidage\\fermer} (p2);
		\draw[<-] (p2) -- node[below] {} ++(0,2cm);

	\end{tikzpicture}
	}%
\end{figure}

On suppose qu'à l'état initial le chariot est à l'arrêt en P2, la trappe est fermée et le tapis chargeur est à l'arrêt. Il faudra compléter ce modèle par l'arrêt d'urgence.

