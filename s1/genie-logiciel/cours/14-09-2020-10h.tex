\section{14 Septembre 2020}

Le prof a décidé de faire le TD2 en CM (roue libre).

\subsection{TD2 - Exercice 2}

\subsection{UML}

L'UML, pour Unified Modeling Language, est une méthode de modélisation orientée objet. Il utilise 14 types de diagrammes pour modéliser un système, de l'analyse des besoins jusqu'à son implémentation.

\subsubsection{Cas d'utilisation}

C'est un ensemble de scénarios partageant le même but. On distingue le scénario nominal des scénarios alternatifs des erreurs.

Des relations peuvent exister entre les différents cas d'utilisation :

\begin{enumerate}
	\item \textbf{L'inclusion}. si A fait appel à B pendant son exécution, alors on dit que A "includes" B.
	\item \textbf{L'extension}. Si B peut faire appel à A, alors on dit que A "extends" B.
	\item \textbf{L'héritage}
\end{enumerate}

\subsubsection{Diagramme de séquence}

