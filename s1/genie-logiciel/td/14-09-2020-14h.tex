\section{14 Septembre 2020 - 14h}

Planche 1 - Seulement envoyée par mail.

\section{Exercice 1}

Les phases d'un modèle de cycle de vie sont :

\begin{itemize}
\tightlist
\item spécification
\item conception
\item codage (implémentation)
\item test et livraison
\end{itemize}

On peut aussi ajouter faisabilité et planification.

Un modèle de cycle de vie mentionne une structure (logique) entre ses phases. Diviser le développement logiciel en différentes phases permet de \textbf{mieux superviser} l'avancement du projet.

Chaque phase est découpée en tâches et sous-tâches. Ceci permet d'identifier les librables (documents et code) de chaque phase.

\section{Exercice 2}

Ordre :

\begin{itemize}
\tightlist
\item étude du marché
\item organisation du projet
\item estimation des coûts
\item synthèse des exigences
\item spécification des exigences
\item conception de haut niveau
\item conception de bas niveau
\item implémentation
\item tests unitaires
\item tests système (note : tests fait par le développeur)
\item tests d'acceptation (note : tests fait par le client)
\end{itemize}

\section{Exercice 3}

\subsection{Faisabilité}

Cahier des charges

\subsection{Planification}

Plan d'assurance qualité

\subsection{Spécification}

Manuel d'utilisateur préliminaire
Plan de test (validation)

\subsection{Conception}

Conception architecturale
Spécification des modules (conception détaillée)
Plan de test

\subsection{Codage}

Code source
Documentation de code

\subsection{Tests}

Rapport des tests

\subsection{Livraison}

Manuel utilisateur final

